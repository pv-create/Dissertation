%\pdfminorversion=7
%\begin{filecontents*}[overwrite]{\jobname.xmpdata}
%\Title{On the de Rham complex over weighted Holder spaces}
%\Author{Ksenia Gagelgans}
%\Language{ru}
%\end{filecontents*}
\documentclass[a4paper]{article}
\usepackage[14pt]{extsizes}
\usepackage[utf8]{inputenc}
\usepackage[english,russian]{babel}
\usepackage{color}
\usepackage{graphicx}
\usepackage{amsmath,amssymb,amsfonts,amsthm,enumerate,latexsym,cite}

\usepackage[left=25mm, right=10mm, top=2cm, bottom=2cm]{geometry} 
%\usepackage[unicode, pdftex, %pdfstartview=FitBH, %colorlinks
%pdfview=FitH]{hyperref}

\usepackage{colorprofiles}
%\usepackage[a-2b,mathxmp]{pdfx}[2018/12/22]

\baselineskip 25pt
\renewcommand{\baselinestretch}{1.5}

\pagenumbering{arabic}

\renewcommand{\theequation}{\thesection.\thesubsection.\arabic{equation}}
\numberwithin{equation}{subsection}

\headsep 18pt%расстояние от верхнего колонтитула
\headheight 25pt%высота колонтитула
\makeatletter
\renewcommand{\@oddfoot}{}%нет нижних колонтитулов
\renewcommand{\@evenfoot}{}%
\makeatother \makeatletter
\renewcommand{\@oddhead}% колонтитулы нечетных и на первой странице
{\raisebox{0pt}[\headheight][0pt]{% номера страниц
\vbox{\hbox to\textwidth{\strut\hfil
\thepage\hfil}}}%
}
\renewcommand{\@evenhead}%
{\raisebox{0pt}[\headheight][0pt]{% номера страниц
\vbox{\hbox to\textwidth{\hfil\strut\thepage\hfil}}}%
} \makeatother

\newtheorem{theorem}{\quad Теорема}[subsection]
\newtheorem{lemma}[theorem]{\quad Лемма}
\newtheorem{corollary}[theorem]{\quad Следствие}
\newtheorem{proposition}[theorem]{\quad Предложение}
\newtheorem{definition}[theorem]{\quad Определение}
\newtheorem{question}[theorem]{\quad Вопрос}
\newtheorem{example}[theorem]{\quad Пример}
\newtheorem{remark}[theorem]{\quad Замечание}
\numberwithin{theorem}{subsection}

\newcommand{\thm}{\noindent\textbf{Теорема }}
\newcommand{\defi}{\noindent\textbf{Определение }}
\newcommand{\prop}{\noindent\textbf{Предложение }}
\newcommand{\lem}{\noindent\textbf{Лемма }}
\newcommand{\rem}{\noindent\textbf{Замечание }}
\newcommand{\exx}{\noindent\textbf{Пример }}
\newcommand{\cor}{\noindent\textbf{Следствие }}

\DeclareMathOperator{\grad}{grad}
\DeclareMathOperator{\divv}{div}
\DeclareMathOperator{\rot}{rot}

\begin{document}

%\renewcommand{\contentsname}{Содержание}
{
	\small
	\
	\thispagestyle{empty}

	\centerline{\textbf{ МОСКОВСКИЙ ГОСУДАРСТВЕННЫЙ УНИВЕРСИТЕТ
	ИМЕНИ М.В.ЛОМОНОСОВА}}

	\medskip

	\centerline{\sc  Механико-математический факультет}
  
	\medskip

	\centerline{\sc <<Кафедра дифференциальной геометрии>>}

	\vskip 7em

	{\hfill на правах рукописи\phantom{asdfgas}

	\vskip 1em

	%=============

	%Здесь нужно изменить

	\hskip 12.5cm
	%=============

	\vskip 2em

	\centerline{\sc\normalsize Вилков Павел Юрьевич}

	\bigskip
	\bigskip

	\centerline{\bf\normalsize НАЗВАНИЕ  }
	%\centerline{\bf\normalsize О КОМПЛЕКСЕ ДЕ РАМА НАД ВЕСОВЫМИ }
	\centerline{\bf\normalsize РАБОТЫ }
	%\centerline{\bf\normalsize ПРОСТРАНСТВАМИ ГЁЛЬДЕРА НА МНОГООБРАЗИИ}


	\bigskip
	\bigskip

	\centerline{1.1.1 --- Математический анализ, дифференциальные уравнения}
  \centerline{(физико-математические науки)} 
	\bigskip
	\bigskip

	\centerline{Диссертация на соискание ученой степени}
	\centerline{кандидата физико-математических наук}

	\bigskip
	\bigskip
	\vskip 6em

	\hfill Научный руководитель:

	\hfill доктор физ.-мат. наук, профессор,

	\hfill Александр Сергеевич Скворцов

	\vskip 7em

	\centerline{Москва\,--\,202x}
}
}

\newpage
\normalsize
\setcounter{page}{2}
\renewcommand\contentsname{\centering Содержание}
\tableofcontents

\newpage
\section*{\centering Введение}
\addcontentsline{toc}{section}{Введение}   

\newpage
%\captionlabeldelim
%\renewcommand{\captionlabeldelim}{.}
\renewcommand\refname{\centering Список использованной литературы}
\begin{thebibliography}{99}
\addcontentsline{toc}{section}{Список использованной литературы}

\end{thebibliography}
\end{document}