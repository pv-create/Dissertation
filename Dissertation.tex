%\pdfminorversion=7
%\begin{filecontents*}[overwrite]{\jobname.xmpdata}
%\Title{On the de Rham complex over weighted Holder spaces}
%\Author{Ksenia Gagelgans}
%\Language{ru}
%\end{filecontents*}
\documentclass[a4paper]{article}
\usepackage[14pt]{extsizes}
\usepackage[utf8]{inputenc}
\usepackage[english,russian]{babel}
\usepackage{color}
\usepackage{graphicx}
\usepackage{amsmath,amssymb,amsfonts,amsthm,enumerate,latexsym,cite}

\usepackage[left=25mm, right=10mm, top=2cm, bottom=2cm]{geometry} 
%\usepackage[unicode, pdftex, %pdfstartview=FitBH, %colorlinks
%pdfview=FitH]{hyperref}

\usepackage{colorprofiles}
%\usepackage[a-2b,mathxmp]{pdfx}[2018/12/22]

\baselineskip 25pt
\renewcommand{\baselinestretch}{1.5}

\pagenumbering{arabic}

\renewcommand{\theequation}{\thesection.\thesubsection.\arabic{equation}}
\numberwithin{equation}{subsection}

\headsep 18pt%расстояние от верхнего колонтитула
\headheight 25pt%высота колонтитула

% Новое определение колонтитулов
\makeatletter
\renewcommand{\@oddfoot}{\hfil\thepage\hfil}
\renewcommand{\@evenfoot}{\hfil\thepage\hfil}
\renewcommand{\@oddhead}{}
\renewcommand{\@evenhead}{}
\makeatother

\newtheorem{theorem}{\quad Теорема}[subsection]
\newtheorem{lemma}[theorem]{\quad Лемма}
\newtheorem{corollary}[theorem]{\quad Следствие}
\newtheorem{proposition}[theorem]{\quad Предложение}
\newtheorem{definition}[theorem]{\quad Определение}
\newtheorem{question}[theorem]{\quad Вопрос}
\newtheorem{example}[theorem]{\quad Пример}
\newtheorem{remark}[theorem]{\quad Замечание}
\numberwithin{theorem}{subsection}

\newcommand{\thm}{\noindent\textbf{Теорема }}
\newcommand{\defi}{\noindent\textbf{Определение }}
\newcommand{\prop}{\noindent\textbf{Предложение }}
\newcommand{\lem}{\noindent\textbf{Лемма }}
\newcommand{\rem}{\noindent\textbf{Замечание }}
\newcommand{\exx}{\noindent\textbf{Пример }}
\newcommand{\cor}{\noindent\textbf{Следствие }}

\DeclareMathOperator{\grad}{grad}
\DeclareMathOperator{\divv}{div}
\DeclareMathOperator{\rot}{rot}

\begin{document}

%\renewcommand{\contentsname}{Содержание}
{
    \small
    \
    \thispagestyle{empty}

    \centerline{\textbf{ МОСКОВСКИЙ ГОСУДАРСТВЕННЫЙ УНИВЕРСИТЕТ
    ИМЕНИ М.В.ЛОМОНОСОВА}}

    \medskip

    \centerline{\sc  Механико-математический факультет}
  
    \medskip

    \centerline{\sc <<Кафедра дифференциальной геометрии>>}

    \vskip 7em

    {\hfill на правах рукописи\phantom{asdfgas}

    \vskip 1em

    %=============

    %Здесь нужно изменить

    \hskip 12.5cm
    %=============

    \vskip 2em

    \centerline{\sc\normalsize Вилков Павел Юрьевич}

    \bigskip
    \bigskip

    \centerline{\bf\normalsize НАЗВАНИЕ  }
    %\centerline{\bf\normalsize О КОМПЛЕКСЕ ДЕ РАМА НАД ВЕСОВЫМИ }
    \centerline{\bf\normalsize РАБОТЫ }
    %\centerline{\bf\normalsize ПРОСТРАНСТВАМИ ГЁЛЬДЕРА НА МНОГООБРАЗИИ}


    \bigskip
    \bigskip

    \centerline{1.1.1 --- Математический анализ, дифференциальные уравнения}
    \centerline{(физико-математические науки)} 
    \bigskip
    \bigskip

    \centerline{Диссертация на соискание ученой степени}
    \centerline{кандидата физико-математических наук}

    \bigskip
    \bigskip
    \vskip 6em

    \hfill Научный руководитель:

    \hfill доктор физ.-мат. наук, профессор,

    \hfill Шлапунов Александр Анатольевич

    \vskip 6em

    \centerline{Москва\,--\,202x}
}
}

\newpage
\normalsize
\renewcommand\contentsname{\centering Содержание}
\setcounter{page}{2}
\tableofcontents

\newpage
\section*{\centering Введение}
\addcontentsline{toc}{section}{Введение}   

\newpage

\section{Предварительные сведенья}
\subsection{Дифференциальные опреаторы}
\subsubsection{Эллиптические опреаторы}
\begin{equation}\label{eqsys}
   \mathcal{L}u = g
\end{equation}
$u = (u_1(x), \dots , u_n(x))^T$ $g = (g_1(x), \dots , g_n(x))^T$

Предположим, что существуют два набора целых чисел (целых векторов)
\begin{equation}
   s = (s_1, \dots, s_n) \text{ и } t = (t_1, \dots, t_n)
\end{equation}
удовлетворяющих условию $\text{deg}l_{i,j}\leq s_i + t_j$ при всех $i, j$ для которых $l_{i,j}\ne 0$ при этом 
$l_{i,j} = 0$ если $s_i+t_j < 0$

Главной частью такого опреатора назовем матрицу вида 
\begin{equation}
   \tilde{\mathcal{L}}{s,t}(x,D) =
   \begin{pmatrix}
   \tilde{l}_{11}^{s,t}(x,D) & \cdots & \tilde{l}_{1N}^{s,t}(x,D) \\
   \vdots & \ddots & \vdots \\
   \tilde{l}{N1}^{s,t}(x,D) & \cdots & \tilde{l}_{NN}^{s,t}(x,D)
   \end{pmatrix},
\end{equation}

\begin{equation}
   \tilde{l}{ij}^{s,t}(x,D) =
   \begin{cases}
   0, \text{если } \deg l{ij}(x,D) < s_i + t_j \text{ или } l_{ij}(x,D) = 0, \\
   \sum_{|\alpha|=s_i+t_j} a_{ij}^{\alpha}(x)D^{\alpha},  \text{если } \deg l_{ij}(x,D) = s_i + t_j.
   \end{cases}
\end{equation}

\begin{definition}
   Система уравнений \ref{eqsys} называется эллиптической по Дуглису–Ниренбергу в области $\Omega$,
   если для любого $x\in\Omega$ и для любого $\xi\in\mathbb{R}\setminus\{0\}$ выполняется условие 
   \begin{equation*}
      \text{det}\tilde{\mathcal{L}}{s,t}(x,\xi) \ne 0
   \end{equation*}
\end{definition}

\begin{example}
   \begin{equation}
      \begin{pmatrix}
         -\Delta& \nabla \\
         -\text{div} &  0 \\
         \end{pmatrix},
   \end{equation}
   Очевидно, оператор не является эллептическим по петровсокму
   Рассмотрим вектора $s = (1, 0, 0)$ и $t = (1, 0, 0)$, тогда главный символ сможем записать в виде 
   \begin{equation}
      \begin{pmatrix}
         -|\xi|^2& \xi \\
         -\xi &  0 \\
         \end{pmatrix}\ne 0
   \end{equation}
   Аналогично можно рассмотреть оператор 
   \begin{equation*}
      \begin{pmatrix}
         -\mathcal{L}& \nabla \\
         -\text{div} &  0 \\
         \end{pmatrix},
   \end{equation*}
   где $\mathcal{L}$ -- стационарный оператор Ламе
\end{example}


\newpage

\renewcommand\refname{\centering Список использованной литературы}
\begin{thebibliography}{99}
\addcontentsline{toc}{section}{Список использованной литературы}

\end{thebibliography}
\end{document}